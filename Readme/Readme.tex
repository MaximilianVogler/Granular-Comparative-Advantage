
\documentclass[12pt,a4paper,oneside]{article}


%%%%%%%%%%%%%%%%%%%%%%%%%
%Preamble
%%%%%%%%%%%%%%%%%%%%%%%%%


\newcommand{\shortauthor}{Maximilian J. Vogler}
\usepackage{styleps}
\usepackage{caption}
\usepackage{float}
\usepackage{subcaption}
\usepackage{ulem}

%Define pagestyle
\pagestyle{fancy}
\setlength{\headheight}{37.5pt}
\fancyhf{}
%\renewcommand\headrule{
%\begin{minipage}{1\textwidth}
%\hrule width \hsize \kern 0.5mm \hrule width \hsize height 1pt 
%\end{minipage}}%
%\lhead{\bf{\shortauthor}}      
%\rhead{\sc{\runtitle}}
%\chead{\includegraphics[width=9pt,height=11pt]{Plogo.png}}
\fancyfoot[C]{\thepage}


%%%%%%%%%%%%%%%%%%%%%%%%%
%Body
%%%%%%%%%%%%%%%%%%%%%%%%%

\begin{document}

\title{\textsc{Readme for the Code for "Granular Comparative Advantage"}}
\author{Maximilian J. Vogler}
\date{19 December 2018}


\makeatletter
\let\runauthor\@author
\let\runtitle\@title    
\makeatother

\maketitle
\tableofcontents
\noindent

\newpage

\section{Data}

\subsection{cdshares\_v3.csv}
This data file contains the Cobb-Douglas shares $\alpha_z$ for 119 4-digit manufacturing industries in France.
\subsection{Data\_Moments.csv}
This file contains the target moment for the GMM estimation procedure, in particular, the 15 data moments from table 2.
\subsection{Data\_Moments\_Explanation.csv}
This file contains the same values as "Data\_Moments.csv" plus explanations for which moment each line corresponds to.
\subsection{cdshares\_and\_theta\_v3.csv}
This file contains the Cobb-Douglas shares  $\alpha_z$  (as "cdshares\_v3.csv") and the Pareto shapes $\kappa_z$ for 119 4-digit manufacturing industries in France. 
\subsection{datamoments\_56.csv}
This file contains additional data moments beyond those used for the GMM estimation.
\subsection{estimated\_vmu\_nopareto.mat}
This file contains the results of the baseline estimation procedure.


\newpage

\section{Baseline}
\subsection{Auxiliary Functions}
These functions will never have to be run individually. They will be called by the "Main Files" explained below.
\subsubsection{vMU\_markup.m}
\begin{enumerate}
\item Purpose: Compute variable markup depending on the kind of competition (i.e. Bertrand or Cournot)/
\item Input: $\sigma$, market shares, kind of competition: BER==1 corresponds to Bertrand and BER==0 to Cournot
competition.
\item Output: Variable Markups
\end{enumerate}

\subsubsection{moment\_stats.m}
\begin{enumerate}
\item Purpose: Compute mean and standard deviation of input
\item Input: vector 
\item Output: mean and standard deviation of all vector elements
\end{enumerate}
\subsubsection{Inner\_Loops.m}
\begin{enumerate}
\item Purpose: Run the two innermost loops (Atkeson-Burstein and entry-exit loops) 
\item Input: $\sigma, \theta, F, \tau, \alpha_z, \varphi_{z,j}, T_z, T_z^*, w, w^*, Y_0, Y_0^*$, indicator for variable markups, indicator for kind of competition
\item Output: A range of output variables that are needed in future code
\item Calls on: vMU\_markup.m
\end{enumerate}
\subsubsection{PEreplication\_vectorized.m}
\begin{enumerate}
\item Purpose: Solves the model in partial equilibrium, that is run the two innermost loops separately for large and small sectors.
\item Input: $\sigma, \theta, F, \tau, \alpha_z, \{\varphi_{z,j}, T_z, T_z^*\}_{\text{small}}^{\text{large}}, w, w^*, Y_0, Y_0^*$, indicator for small sectors, indicator for variable markups, indicator for kind of competition
\item Output: A range of output variables that are needed in future code
\item Calls on: Inner\_Loops
\end{enumerate}
\subsubsection{GEreplication\_vectorized.m}
\begin{enumerate}
\item Purpose: Solves the model in general equilibrium with $\frac{w}{w^*}$ fixed. 
\item Input: $\sigma, \theta, F, \tau, \alpha_z, \{\varphi_{z,j}, T_z, T_z^*\}_{\text{small}}^{\text{large}}, w, w^*, Y_0, Y_0^*$, indicator for small sectors, indicator for variable markups, indicator for kind of competition
\item Output: $Y, Y^*, L^*$ and some output needed to generate moments in future code
\item Calls on: PEreplication\_vectorized.m
\end{enumerate}
\subsubsection{Moments.m}
\begin{enumerate}
\item Purpose: Computes the 15 target moment from table 2 that are required for estimation.
\item Input: Some of the previous output
\item Output: vector of 15 target moments
\item Calls on: moment\_stats
\end{enumerate}
\subsubsection{Loss\_Function.m}
\begin{enumerate}
\item Purpose: Computes the loss for the given set of input parameters.
\item Input: Parameters 
\item Output: loss value
\item Calls on: GEreplication\_vectorized.m and Moments.m
\end{enumerate}


\newpage

\subsection{Main Files}
These files will need to be run to fully estimate the model. In particular, one needs to run "Estimation.m" first, followed by "Grid\_Optimization.m" and lastly "Local\_Minimization.m".

\subsubsection{Estimation.m}
\begin{enumerate}
\item Purpose: This code performs the estimation routine as outlined in the appendix, steps 1-3.
\item Input: cdshares\_v3.csv
\item Output: estimation\_seed1\_grid6: a file that includes both the chosen parameters and the 15 target moments at those parameters.
\item Calls on: GEreplication\_vectorized.m, Moments.m
\end{enumerate}
\subsubsection{Grid\_Optimization.m}
\begin{enumerate}
\item Purpose: Find the 20 best points from the inputted data file as in steps 4-5 of the estimation procedure.
\item Input: Data\_Moments.csv, estimation\_seed1\_grid6
\item Output: GridOptimization\_seed1\_grid6: a file that contains the final 20 best parameter vectors
\end{enumerate}
\subsubsection{Local\_Minimization.m}
\begin{enumerate}
\item Purpose: This code performs local minimization around the 20 best grid points as in step 6 of the estimation procedure.
\item Input: GridOptimization\_seed1\_grid6, Data\_Moments.csv
\item Output: local_min_seed1_grid6: Initial parameter vectors, locally minimized parameter vectors and loss values
\item Calls on: Loss\_Function.m 
\end{enumerate}

\section{Theta Heterogeneity}

\subsection{Auxiliary Files}

\subsubsection{vMU\_markup.m}
Exactly the same as the above file in the baseline model.
\subsubsection{Inner\_Loops\_thetas.m}
\begin{enumerate}
\item Purpose: Run the two innermost loops (Atkeson-Burstein and entry-exit loops) for the case of theta heterogeneity.
\item Input: $\sigma, F, \tau, \alpha_z, \theta_z, \varphi_{z,j}, T_z, T_z^*, w, w^*, Y_0, Y_0^*$, indicator for variable markups, indicator for kind of competition, and "paretobn", which is a bound for the computation of the Pareto shape.
\item Output: A range of output variables that are needed in future code
\item Calls on: vMU\_markup.m
\end{enumerate}
\subsubsection{PEreplication\_thetas.m}
\begin{enumerate}
\item Purpose: Solves the model in partial equilibrium with heterogeneous thetas, that is it runs the two innermost loops separately for large and small sectors.
\item Input: $\sigma, \theta, F, \tau, \{\varphi_{z,j}, T_z, T_z^*\}_{\text{small}}^{\text{large}}, w, w^*, Y_0, Y_0^*$, indicator for small sectors, indicator for variable markups, indicator for kind of competition, "paretobn", a matrix containing $\alpha_z$ and $\theta_z$
\item Output: A range of output variables that are needed in future code
\item Calls on: Inner\_Loops\_thetas
\end{enumerate}

\subsubsection{GEreplication\_thetas.m}
\begin{enumerate}
\item Purpose: Solves the model in general equilibrium with $\frac{w}{w^*}$ fixed for the case of theta heterogeneity. 
\item Input: $\sigma, \theta, F, \tau, \{\varphi_{z,j}, T_z, T_z^*\}_{\text{small}}^{\text{large}}, w, w^*, Y_0, Y_0^*$, indicator for small sectors, indicator for variable markups, indicator for kind of competition, "paretobn", a matrix containing $\alpha_z$ and $\theta_z$
\item Output: $Y, Y^*, L^*$ and some output needed to generate moments in future code
\item Calls on: PEreplication\_thetas.m
\end{enumerate}

\subsection{Main File}

\subsubsection{Het\_Theta\_Run.m}
\begin{enumerate}
\item Purpose: This code needs to be executed to run the model for the case of theta heterogeneity and saves the results needed for table 3.
\item Input: estimated\_vmu\_nopareto.mat, cdshares\_and\_theta\_v3.csv, datamoments\_56.csv
\item Output: 5 mat-files containing the results for the 5 columns of table 3.
\item Calls on: GEreplication\_thetas.m
\end{enumerate}




\end{document}
